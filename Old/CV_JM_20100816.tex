%%%%%%%%%%%%%%%%%%%%%%%%%%%%%%%%%%%%%%%%%%%%%%%%%%%%%%%%%%%%%%%%%%%%%%%%
%%%%%%%%%%%%%%%%%%%%%% Simple LaTeX CV Template %%%%%%%%%%%%%%%%%%%%%%%%
%%%%%%%%%%%%%%%%%%%%%%%%%%%%%%%%%%%%%%%%%%%%%%%%%%%%%%%%%%%%%%%%%%%%%%%%

%%%%%%%%%%%%%%%%%%%%%%%%%%%%%%%%%%%%%%%%%%%%%%%%%%%%%%%%%%%%%%%%%%%%%%%%
%% NOTE: If you find that it says                                     %%
%%                                                                    %%
%%                           1 of ??                                  %%
%%                                                                    %%
%% at the bottom of your first page, this means that the AUX file     %%
%% was not available when you ran LaTeX on this source. Simply RERUN  %%
%% LaTeX to get the ``??'' replaced with the number of the last page  %%
%% of the document. The AUX file will be generated on the first run   %%
%% of LaTeX and used on the second run to fill in all of the          %%
%% references.                                                        %%
%%%%%%%%%%%%%%%%%%%%%%%%%%%%%%%%%%%%%%%%%%%%%%%%%%%%%%%%%%%%%%%%%%%%%%%%

%%%%%%%%%%%%%%%%%%%%%%%%%%%% Document Setup %%%%%%%%%%%%%%%%%%%%%%%%%%%%

% Don't like 10pt? Try 11pt or 12pt
\documentclass[10pt]{article}

% This is a helpful package that puts math inside length specifications
\usepackage{calc}

% Simpler bibsection for CV sections
% (thanks to natbib for inspiration)
\makeatletter
\newlength{\bibhang}
\setlength{\bibhang}{1em}
\newlength{\bibsep}
 {\@listi \global\bibsep\itemsep \global\advance\bibsep by\parsep}
\newenvironment{bibsection}
    {\minipage[t]{\linewidth}\list{}{%
        \setlength{\leftmargin}{\bibhang}%
        \setlength{\itemindent}{-\leftmargin}%
        \setlength{\itemsep}{\bibsep}%
        \setlength{\parsep}{\z@}%
        }}
    {\endlist\endminipage}
\makeatother

% Layout: Puts the section titles on left side of page
\reversemarginpar

%
%         PAPER SIZE, PAGE NUMBER, AND DOCUMENT LAYOUT NOTES:
%
% The next \usepackage line changes the layout for CV style section
% headings as marginal notes. It also sets up the paper size as either
% letter or A4. By default, letter was used. If A4 paper is desired,
% comment out the letterpaper lines and uncomment the a4paper lines.
%
% As you can see, the margin widths and section title widths can be
% easily adjusted.
%
% ALSO: Notice that the includefoot option can be commented OUT in order
% to put the PAGE NUMBER *IN* the bottom margin. This will make the
% effective text area larger.
%
% IF YOU WISH TO REMOVE THE ``of LASTPAGE'' next to each page number,
% see the note about the +LP and -LP lines below. Comment out the +LP
% and uncomment the -LP.
%
% IF YOU WISH TO REMOVE PAGE NUMBERS, be sure that the includefoot line
% is uncommented and ALSO uncomment the \pagestyle{empty} a few lines
% below.
%

%% Use these lines for letter-sized paper
\usepackage[paper=letterpaper,
            %includefoot, % Uncomment to put page number above margin
            marginparwidth=1.2in,     % Length of section titles
            marginparsep=.05in,       % Space between titles and text
            margin=1in,               % 1 inch margins
            includemp]{geometry}

%% Use these lines for A4-sized paper
%\usepackage[paper=a4paper,
%            %includefoot, % Uncomment to put page number above margin
%            marginparwidth=30.5mm,    % Length of section titles
%            marginparsep=1.5mm,       % Space between titles and text
%            margin=25mm,              % 25mm margins
%            includemp]{geometry}

%% More layout: Get rid of indenting throughout entire document
\setlength{\parindent}{0in}

%% This gives us fun enumeration environments. compactitem will be nice.
\usepackage{paralist}

%% Reference the last page in the page number
%
% NOTE: comment the +LP line and uncomment the -LP line to have page
%       numbers without the ``of ##'' last page reference)
%
% NOTE: uncomment the \pagestyle{empty} line to get rid of all page
%       numbers (make sure includefoot is commented out above)
%
\usepackage{fancyhdr,lastpage}
\pagestyle{fancy}
%\pagestyle{empty}      % Uncomment this to get rid of page numbers
\fancyhf{}\renewcommand{\headrulewidth}{0pt}
\fancyfootoffset{\marginparsep+\marginparwidth}
\newlength{\footpageshift}
\setlength{\footpageshift}
          {0.5\textwidth+0.5\marginparsep+0.5\marginparwidth-2in}
\lfoot{\hspace{\footpageshift}%
       \parbox{4in}{\, \hfill %
                    \arabic{page} of \protect\pageref*{LastPage} % +LP
%                    \arabic{page}                               % -LP
                    \hfill \,}}

% Finally, give us PDF bookmarks
\usepackage{color,hyperref}
\definecolor{darkblue}{rgb}{0.0,0.0,0.3}
\hypersetup{colorlinks,breaklinks,
            linkcolor=darkblue,urlcolor=darkblue,
            anchorcolor=darkblue,citecolor=darkblue}

%%%%%%%%%%%%%%%%%%%%%%%% End Document Setup %%%%%%%%%%%%%%%%%%%%%%%%%%%%


%%%%%%%%%%%%%%%%%%%%%%%%%%% Helper Commands %%%%%%%%%%%%%%%%%%%%%%%%%%%%

% The title (name) with a horizontal rule under it
%
% Usage: \makeheading{name}
%
% Place at top of document. It should be the first thing.
\newcommand{\makeheading}[1]%
        {\hspace*{-\marginparsep minus \marginparwidth}%
         \begin{minipage}[t]{\textwidth+\marginparwidth+\marginparsep}%
                {\large \bfseries #1}\\[-0.15\baselineskip]%
                 \rule{\columnwidth}{1pt}%
         \end{minipage}}

% The section headings
%
% Usage: \section{section name}
%
% Follow this section IMMEDIATELY with the first line of the section
% text. Do not put whitespace in between. That is, do this:
%
%       \section{My Information}
%       Here is my information.
%
% and NOT this:
%
%       \section{My Information}
%
%       Here is my information.
%
% Otherwise the top of the section header will not line up with the top
% of the section. Of course, using a single comment character (%) on
% empty lines allows for the function of the first example with the
% readability of the second example.
\renewcommand{\section}[2]%
        {\pagebreak[2]\vspace{1.3\baselineskip}%
         \phantomsection\addcontentsline{toc}{section}{#1}%
         \hspace{0in}%
         \marginpar{
         \raggedright \scshape #1}#2}

% An itemize-style list with lots of space between items
\newenvironment{outerlist}[1][\enskip\textbullet]%
        {\begin{itemize}[#1]}{\end{itemize}%
         \vspace{-.6\baselineskip}}

% An environment IDENTICAL to outerlist that has better pre-list spacing
% when used as the first thing in a \section
\newenvironment{lonelist}[1][\enskip\textbullet]%
        {\vspace{-\baselineskip}\begin{list}{#1}{%
        \setlength{\partopsep}{0pt}%
        \setlength{\topsep}{0pt}}}
        {\end{list}\vspace{-.6\baselineskip}}

% An itemize-style list with little space between items
\newenvironment{innerlist}[1][\enskip\textbullet]%
        {\begin{compactitem}[#1]}{\end{compactitem}}

% To add some paragraph space between lines.
% This also tells LaTeX to preferably break a page on one of these gaps
% if there is a needed pagebreak nearby.
\newcommand{\blankline}{\quad\pagebreak[2]}

% 

%%%%%%%%%%%%%%%%%%%%%%%% End Helper Commands %%%%%%%%%%%%%%%%%%%%%%%%%%%

%%%%%%%%%%%%%%%%%%%%%%%%% Begin CV Document %%%%%%%%%%%%%%%%%%%%%%%%%%%%

\begin{document}
\makeheading{John Muschelli}

\section{Contact Information}
%
% NOTE: Mind where the & separators and \\ breaks are in the following
%       table.
%
% ALSO: \rcollength is the width of the right column of the table
%       (adjust it to your liking; default is 1.85in).
%
\newlength{\rcollength}\setlength{\rcollength}{1.8in}%
%
\begin{tabular}[t]{@{}p{\textwidth-\rcollength}p{\rcollength}}
                            121 S Ann Street           & \textit{Cell:} (610) 291-7685 \\

  Baltimore, ~MD~21231~USA         & \textit{E-mail:}
\href{mailto:jmuschel@jhsph.edu}{jmuschel@jhsph.edu}\\
\textit{WWW:}
\href{http://www.biostat.jhsph.edu/people/student/muschelli.shtml}{http://www.biostat.jhsph.edu/people/student/muschelli.shtml} & \\
\end{tabular}

\section{Research Interests}
%
Clinical trials, brain image analysis, resting state connectivity, iterative processes.
\section{Education}
%
\href{http://www.biostat.jhsph.edu/}{\textbf{Johns Hopkins School of Public Health}},
Baltimore, Maryland USA
\begin{outerlist}
\item[] ScM,
        \href{http://biostat.jhsph.edu/}
             {Biostatistics} (May 2010)
        \begin{innerlist}
        \item Thesis Topic: \emph{A Critical Review of HRF Derivatives in SPM for Functional Neuroimaging}
        \item Adviser:
              \href{http://www.biostat.jhsph.edu/~bcaffo/}
                   {Professor Brian Caffo}
        \item Area of Study: fMRI brain image data analysis\\
        \end{innerlist}
\end{outerlist}
        
\href{http://www.scranton.edu/}{\textbf{The University of Scranton}},
Scranton, Pennsylvania USA
\begin{outerlist}

\item[] B.S.,
        \href{http://math.scranton.edu/}
             {Biomathematics and Neuroscience}
            (Summa Cum Laude) (May 2008)
        \begin{innerlist}
        \item Adviser:
              \href{http://academic.scranton.edu/department/math/Faculty/inf_jasi.html}
                   {Professor Jakub Jasinski}
        \item Area of Study: Biomathematics\\
        \item Adviser:
              \href{http://academic.scranton.edu/faculty/cannon/}
                   {Professor J.~Timothy Cannon}
        \item Area of Study: Neuroscience
        \end{innerlist}


\end{outerlist}

\section{Awards}
%
\href{http://www.scranton.edu}{The University of Scranton}
\begin{innerlist}
\item Presidential Scholar (Full Tuition Scholarship), 2004-2008
\item Dean's List, 2004-2008
\item Alpha Lambda Delta, 2004
\item Alpha Sigma Nu, 2008

\end{innerlist}

\blankline


\section{Academic Experience}
\href{http://www.jhsph.edu}{\textbf{Johns Hopkins School of Public Health}},
Baltimore, Maryland USA
\begin{outerlist}

\item[] \textit{Teaching Assistant}%
    \hfill \textbf{September 2009 to May 2010}
    \begin{innerlist}
        \item Biostatistics 651-4: Methods in Biostatistics I-IV
        \begin{innerlist}
            \item Fall~2009-May~2010
            \item Responsible for 1~hour lab helping session, and grading homework and tests.
            \end{innerlist}~

        \item Biostatistics 613-4: Data Analysis Workshop I-II
        \begin{innerlist}
            \item January~2009
             \item Responsible for 4~hour lab and teaching session, helping students to get comfortable with Stata programming and interpreting results from output.

        \end{innerlist}~
        

        
    \end{innerlist}



\item[] \textit{Tutor}%
        \hfill \textbf{March 2005 to May 2008}
\begin{innerlist}
\item Tutored students regarding mathematics, statistics and other science courses.
\end{innerlist}

\item[] \textit{Undergraduate Student}%
        \hfill \textbf{September 2004 to June 2008}
        
\end{outerlist}

\section{Professional Experience}
%
\href{http://www.biostat.jhsph.edu/consult/}{\textbf{Johns Hopkins Biostatistics Center (JHBC)}},
Baltimore, Maryland USA
\begin{outerlist}

\item[] \textit{Junior Statistical Consultant}%
        \hfill \textbf{January 2009 to Present}
\begin{innerlist}
\item Collaborated on statistical projects with senior consultants.
\item Report writing and analyzing data using statistical software: R, Stata.
\item Cleaning and checking quality of data.
\end{innerlist}

\end{outerlist}

\blankline

\href{http://www.analysisandinference.com/}{\textbf{Analysis \&Inference }},
Springfield, Pennsylvania USA
\begin{outerlist}

\item[] \textit{Statistical Assistant}%
        \hfill \textbf{June 2008 to August 2008}
\begin{innerlist}
\item Cooperated on statistical projects and conferenced with clients about possible analysis options.
\item Report writing of analyses: Stata
\item Data cleaning
\end{innerlist}


\end{outerlist}

\blankline

\textbf{\href{http://www2.dupont.com/DuPont_Home/en_US/index.html/} Dupont},
Stine-Haskell Laboratory, Wilmington, Delaware USA
\begin{outerlist}
\item[] \textit{Research Interm}%
        \hfill \textbf{June 2007 to August 2007}
\begin{innerlist}
\item Developed lab skills and techniques: cell culturing, making and sterilizing broth media, optical density readings, inoculations, quality control, cell counts, screening for fungicidal properties of compounds.
\end{innerlist}

\end{outerlist}



\section{Service}
\href{http://academic.scranton.edu/organization/edjustice/events.shtml}{F.I.R.S.T. (Freshmen In Reflective Service Together) 2007}
\begin{innerlist}
\item \emph{Leader} Led a service trip as a senior that welcomes incoming freshmen to participate in volunteer service at the University of Scranton.
\item Each team does service at various sites around the city of Scranton, namely Head Start, Little People Day Care, Lackawanna County Health Care Center.
\end{innerlist}

\blankline

Master of Ceremonies, University of Scranton 2005--2008
\begin{innerlist}
\item Midnight Madness is a kickoff campaign for the basketball season and a food-raising activity for local food kitchens.  \item \href{http://www.theofficeconvention.com/}{The Office Convention} was an event where members of the television show The Office came to Scranton, students  entertained crowds in between events on the stages.  
\item Quizzo and Royal Feud are trivia games that are fundraisers for international and domestic service programs.  \end{innerlist}

\blankline

Saint Francis of Assisi Soup Kitchen, 2004--2005
\begin{innerlist}
\item Volunteered working with those running the soup kitchen.
\item Tasks ranged from cleaning dishes to handing out food.
\end{innerlist}


\section{Technical Skills}
%
Software experience in statistics and data processing

\blankline
Statistics:
\href{http://cran.r-project.org/}{\textsc{R}}
        experience: statistical programming, survival analysis, regression modeling, fmri image analysis, packages.

\blankline

\href{http://cran.r-project.org/}{\textsc{R}}
        experience: statistical programming, survival analysis, regression modeling.

\blankline

\href{http://www.sas.com}{\textsc{SAS}}
        experience: basic programming/analysis.

\blankline

\href{http://www.stattransfer.com/}{\textsc{StatTransfer}}
        experience: import/export of multiple data formats.

\blankline


\href{http://www.cs.uiowa.edu/~rlenth/StatWeave/}{\textsc{Statweave}}
        experience: programming multiple languages in one software.

\blankline

\href{http://www.stata.com}{\textsc{Stata}}
        experience: statistical programming survival analysis, regression modeling.

\blankline

Basic Programming: C++, Visual Basic

\blankline

Operating Systems: Microsoft Windows XP/2000/Vista, Mac OS X

\blankline

Applications: \TeX{}, \LaTeX{}, B\textsc{ib}\TeX{}, Microsoft Office, OpenOffice, TeXShop, WinEdt, WinBUGS, MikTeX, Shell.  

\section{Mathematical Experience}
%
Linear Algebra, Real Analysis

\blankline

Probability, Random Variables, and Stochastic Processes, Statistical Inference, Survival Analysis

\blankline

Statistical Programming

\blankline

Functional MRI (fMRI) Imaging Analysis 

\section{Publications}

(In Review)  Newell, D. Shah, M. Wilcox, R. Hansmann, D, Melnychuk, E. Muschelli, J. Hanley, D.\textbf{Minimally invasive evacuation of spontaneous intracerebral hemorrhage using sonothrombolysis.} \emph{Journal of Neurosurgery}

\end{document}

%%%%%%%%%%%%%%%%%%%%%%%%%% End CV Document %%%%%%%%%%%%%%%%%%%%%%%%%%%%%
